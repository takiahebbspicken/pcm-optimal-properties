\documentclass[12pt]{article}
\usepackage[utf8]{inputenc} % Input encoding package
\usepackage[top=25mm, bottom=25mm, left=25mm, right=25mm]{geometry} % Page size and margins

% Math packages
\usepackage{amsmath} % American Mathematical society package cmex10 
\usepackage{amssymb} % Extended symbol collection
\usepackage{amsthm} % For defining theorem-like structures
\usepackage{mathtools} % Enhanced math
\usepackage{mathptmx} % Math times new roman text
\usepackage{bm} % Bold math symbols
\usepackage{mathrsfs} % Math script font

% Figure and table packages
\usepackage{graphicx} % Image management
\usepackage{float} % Define floating figures or tables
\usepackage{adjustbox} % Change size of figures or tables to fill certain sizes
\usepackage{array} % Extended array and tabular environments
\usepackage{subcaption, booktabs, multicol, multirow} % Subfigures
\usepackage{hhline} % Better horizonal lines in tabular and array environments
\usepackage{caption} % Caption customization in figure and table floating environments

% Spacing and styling
\usepackage{setspace} % Single spacing envrionments
\usepackage{parskip} % Paragraph spacing
\usepackage{lscape} % Put parts of document in landscape orientation
\usepackage{titlesec} % Title styling
\usepackage{paralist} % Can use enumerate and itemize environements in paragraphs
\usepackage{color} % Colored text
\usepackage[colorlinks, citecolor=black, linkcolor=black, linktocpage=true]{hyperref} % Cross referencing

% Environment packages
\usepackage{enumerate} % Numbered lists
\usepackage{listings} % Typeset code
\usepackage[framed,numbered]{mcode} % Typeset matlab code specifically
\usepackage[toc,page]{appendix} % Appendices 
\usepackage[acronym, nopostdot]{glossaries} % Glossary with acronyms and no dots after entries 

% Bibliography packages
\usepackage{cite} % Improved numeric citations

\newenvironment{RomanPagenumber}[1]
{\setcounter{page}{#1}\renewcommand{\thepage}{\Roman{page}}}
{\pagenumbering{arabic}}

\newenvironment{romanPagenumber}[1]
{\setcounter{page}{#1}\renewcommand{\thepage}{\roman{page}}}
{\pagenumbering{arabic}}

\newenvironment{arabicPagenumber}[1]
{\setcounter{page}{#1}\renewcommand{\thepage}{\arabic{page}}}
{\pagenumbering{arabic}}

\newenvironment{AlphPagenumber}[1]
{\setcounter{page}{#1}\renewcommand{\thepage}{\Alph{page}}}
{\pagenumbering{arabic}}

\newenvironment{alphPagenumber}[1]
{\setcounter{page}{#1}\renewcommand{\thepage}{\alph{page}}}
{\pagenumbering{arabic}}

% Packed lists
\newenvironment{packed_enum}{
	\begin{enumerate}
		\setlength{\itemsep}{1pt}
		\setlength{\parskip}{0pt}
		\setlength{\parsep}{0pt}
	}{\end{enumerate}}

\newenvironment{packed_itemize}{
	\begin{itemize}
		\setlength{\itemsep}{1pt}
		\setlength{\parskip}{0pt}
		\setlength{\parsep}{0pt}
	}{\end{itemize}}

\usepackage{fancyhdr, lastpage}
\pagestyle{fancy}
\lhead{}
\rhead{} 
\chead{} 
\lfoot{}
\cfoot{\small \thepage}
\rfoot{}
\renewcommand{\headrulewidth}{0.0pt} 
\renewcommand{\footrulewidth}{0.75pt}

% spacing
\setlength{\parskip}{5pt}
% \setlength{\parskip}{\baselineskip}
\setlength{\parindent}{25pt}
\setlength{\belowcaptionskip}{-10pt}

% \titlespacing*{<command>}{<left>}{<before-sep>}{<after-sep>}
\titlespacing*{\section}{0pt}{7.5pt}{0pt}
\titlespacing*{\subsection}{0pt}{7.5pt}{0pt}
\titlespacing*{\subsubsection}{0pt}{7.5pt}{0pt}
\captionsetup[table]{skip=0pt}  

\date{\today}
%\doublespacing

%\makeglossaries
%\input{glossary.tex}

\begin{document}
\begin{titlepage}
	\pagenumbering{gobble} % Turns page numbering off for title page
	%\maketitle %Use if logo not needed
	\centering
	\vfill
	\Huge{
		\textbf{Optimal Phase-Change-Material Properties for Battery Thermal Management: A Multi-objective Design Optimization Approach} 
	}
	\\ 
	\medskip
	\large{
		Authors: Takiah Ebbs-Picken\textsuperscript{a,}\footnote{Corresponding author, e-mail: takiah.ebbspicken@mail.utoronto.ca}, Carlos M. Da Silva\textsuperscript{a}, and Cristina H. Amon\textsuperscript{a}
	}
	\\
	\medskip
	\small{
		\textsuperscript{a} Department of Mechanical and Industrial Engineering, ATOMS Laboratory, University of Toronto, 5 King’s College Road, Toronto, ON M5S 3G8, Canada
		\\
		
	}
	\vfill
\end{titlepage}

\clearpage % new page

\begin{romanPagenumber}{2} %Roman numeral page numbering before report begins (for abstract, acknoledgements, table of contents, etc.)
	
\section*{Highlights}
\addcontentsline{toc}{section}{Highlights}%
%85 char max, 3 - 5 bullets
\begin{itemize}
	\item XYZ
\end{itemize}

\section*{Abstract}
\addcontentsline{toc}{section}{Abstract}%


\section*{Keywords}
\addcontentsline{toc}{section}{Keywords}%
Battery thermal management; battery cooling; design optimization

\newpage


\tableofcontents %Add table of contents
\clearpage
\addcontentsline{toc}{section}{\protect\numberline{Nomenclature}}
\begin{scriptsize}
	\begin{multicols}{2}
%		\printglossary[type=\acronymtype,title={Nomenclature}, nonumberlist]
%		\printglossary[title=Nomenclature]
	\end{multicols}
\end{scriptsize}

\clearpage
\end{romanPagenumber} %End roman numeral page numbering
\pagenumbering{arabic} %Start regular Numbering

\section{Introduction}
\label{sec:intro}
BATTERY PERFORMANCE AT LOW TEMPERATURES

BATTERY PERFORMANCE AT HIGH TEMPERATURES

PCM SYSTEMS FOR COOLING

PCM SYSTEMS FOR HEATING/HEAT RETENTION

OPTIMIZED PCM SYSTEMS
A PCM's melting process absorbs heat and can be applied to cool batteries at elevated temperatures \cite{Hu_Zheng_Howey_Perez_Foley_Pecht_2020}. 
The PCM solidification process releases heat and can be applied to keep batteries warm at low temperatures \cite{Hu_Zheng_Howey_Perez_Foley_Pecht_2020}.

Passive cooling: High latent heat per unit mass, high thermal conductivity, and high specific heat \cite{Jaguemont_Omar_Bossche_Mierlo_2018}.
Should match operating temperature of the heating or cooling ot the PCM transition temperature \cite{Jaguemont_Omar_Bossche_Mierlo_2018}. 

Passive heating: to delay a PCM solidification process it is desirable to have lower thermal conductivity, greater latent heat, and higher environmental temperature. 
However, larger latent heat resulted in more less uniform cell temperatures \cite{Huo_Rao_2017}. 

Important to balance PCM's ability to keep batteries warm during short stops, ensure temperature uniformity, and allow for fast battery temperature rise after long stops \cite{Hu_Zheng_Howey_Perez_Foley_Pecht_2020}.

Low thermal conductivity PCMs result in larger temperature gradients, while higher thermal conductivities allow for faster heat transfer to batteries reducing warm up times \cite{Hu_Zheng_Howey_Perez_Foley_Pecht_2020}.

Soaking period in cold conditions impacts the performance of PCM systems \cite{Ling_Wen_Zhang_Fang_Xu_2016}. 
For a short soaking time the PCM keeps batteries warm, while after a long soaking time the extra thermal mass of the PCM prevents the battery from self-heating as quickly \cite{Ling_Wen_Zhang_Fang_Xu_2016}. 
In the long term, PCM was found to reduce capacity loss for cold temperature operation, increasing the average temperature compared to systems without PCM \cite{Ling_Wen_Zhang_Fang_Xu_2016}. 

\section{Modeling}
PCM MODELS

COMSOL IMPLEMENTATION (BCs, initial conditions, mesh details, mesh independence, etc.)

PARALLELIZE CFD ANALYSIS AFTER SAMPLE SELECTION (https://www.comsol.com/support/knowledgebase/1001) and (https://www.comsol.com/blogs/how-to-run-simulations-in-batch-mode-from-the-command-line/)

Assumptions:
\begin{itemize}
	\item One way coupling of flow and heat generation - the change in temperature is assumed to have no effect on the flow field
	\item 
\end{itemize}
\section{Optimization}
\label{sec:opt}
Consider a fixed volume of PCM and the thermal material properties as the design variables (latent heat, thermal conductivity, specific heat, phase-change temperature).
Consider bounds on the design variables rooted in reality, based on maximum and minimum property quantities of available PCMs. 
Need to consider specific objective functions and properly formulate:
\begin{enumerate}
	\item Heat up time (time in seconds to reach certain temperature)
	\item Cooling performance (maximum temperature, average temperature)
	\item Temperature uniformity (cell temperature difference, cell temperature standard deviation)
	\item Battery heat retention time (starting after running to the maximum temperature, how long do batteries stay warm for)
\end{enumerate}
The optimization problem is thus formulated as shown in Figure~\ref{eqn:optFormulation}.
\begin{equation}
	\begin{aligned}
		\label{eqn:optFormulation}
		\text{minimize} \qquad & f(\textbf{x}) \\
		\text{with respect to} \qquad & \textbf{x} \\
		\text{subject to} \qquad & h_{i}(\textbf{x}) = 0; \qquad i=1 \ \text{to} \ p \\
		& g_{j}(\textbf{x}) \leq 0; \qquad j=1 \ \text{to} \ m
	\end{aligned}
\end{equation}

\clearpage
%Bibliography
\addcontentsline{toc}{section}{References}
\begin{multicols}{2}
	\begin{scriptsize}
		\begin{singlespace}
			\bibliography{references-pop}
			\bibliographystyle{ieeetr}
		\end{singlespace}
	\end{scriptsize}
\end{multicols}
%	
	
%\clearpage
%\begin{appendices} %Appendices
%\section{Code}
%\label{appendix:code}
%\end{appendices}
\end{document}
